\documentclass[12pt]{article}
\usepackage{amsmath}
\usepackage{amsthm}
\usepackage[hmargin=1in,vmargin=1in]{geometry}
\usepackage[shortlabels]{enumitem}
\usepackage{multicol}
\usepackage{hyperref}
\usepackage{graphicx}
\usepackage[section]{placeins}
\usepackage{amssymb}
\usepackage{listings}
\usepackage{amsmath,tkz-euclide}
\usepackage{tikz}
\usetikzlibrary{arrows.meta, decorations.markings}
\usetkzobj{all}
\newtheorem{theorem}{Theorem}
\newtheorem*{theorem*}{Theorem}
%\usepackage{float}
%\floatplacement{figure}{H}
\newcounter{problem}
\newcounter{solution}
\graphicspath{{/Users/edwardmcdugald/Research/convection_patterns/figs/}{/Users/edwardmcdugald/Research/convection_patterns/code/figs/sh_num_tsts_1018/}}
\usepackage{chngcntr}
\counterwithin*{section}{part}

\parindent 0in
\parskip 1em
\Urlmuskip=0mu plus 1mu

\title{Swift-Hohenberg Numerics - Testing with Chebfun}
\author{Edward McDugald}


\begin{document}
\maketitle
\section{Operator Splitting for SH}
\begin{itemize}
        \item We have the PDE
        \[
            w_t = -(1+\nabla^2)^2w + Rw - w^3.
        \]
    \item
        We can break this into a linear part and a non-linear part,
        \[
            W_t = L(w) + NL(w),
        \]
        with
        \begin{align*}
            L(w) &= -(2\nabla^2+\nabla^4)w\\
            NL(w) &=  (R-1)w- w^3.
        \end{align*}
         \item
        Basic procedure is as follows:
        \begin{enumerate}[(i)]
            \item Let $A = (-\nabla^4-2\nabla^2)$. Consider the pair of PDEs
                \begin{align*}
                    w_t &= Aw\\
                    w_t &= (R-1)w-w^3.
                \end{align*}
            \item
                Handling the linear and nonlinear PDEs separately, we get the relations
                \begin{align*}
                    w(t+\Delta t) &= \Delta t Aw(t)+w(t)\\
                    \implies w(t+\Delta t) &\approx e^{A\Delta t}w(t).
                \end{align*}
                And for the nonlinear part,
                \begin{align*}
                    w(t+\Delta t) &\approx \Delta t \left[(R-1)w(t)-w(t)^3\right]+w(t).
                \end{align*}.
            \item Apply operator splitting (strang splitting) now
            \end{enumerate}
    \end{itemize}
    The code is implemented in python as follows:
    \begin{verbatim}
def non_lin_rhs(w,R):
    return (R-1)*w - w**3

def integrateSH(w0,R,dt,nSteps,L):
     """
     :param w0: initial temperature surface
     :param R: bifurcation parameter- can be a constant, or of same shape as w0
     :param dt: time step length
     :param nSteps: number of time steps to take
     :param L: Length of square over which w0 is defined
     :return w0: time evolution of w0 at time 0+dt*nSteps
     Ideally, the size of w0 is fft friendly, ie 2^n x 2^n
     """
     print("Starting time integration of Swift Hohenberg")
     ny, nx = np.shape(w0)
     print("Dimensions of w0:", nx, ny)
     kx = (2.*np.pi/L)*sp.fft.fftfreq(nx,1./nx)
     ky = (2.*np.pi/L)*sp.fft.fftfreq(ny,1./ny)
     Kx, Ky = np.meshgrid(kx,ky)
     fourierLaplacian = -(Kx**2+Ky**2)
     A = -(fourierLaplacian*fourierLaplacian)-2*fourierLaplacian
     for i in range(0,nSteps):
         if i%100 == 0:
             print("step number:",i)
         w1 = np.real(sp.fft.ifft2(np.exp(A*.5*dt)*sp.fft.fft2(w0)))

         #rk4 version
         #k1 = dt*w1
         #k2 = dt*non_lin_rhs(w1+.5*k1, R)
         #k3 = dt*non_lin_rhs(w1+.5*k2, R)
         #k4 = dt*non_lin_rhs(w1+k3, R)
         #w2 = (k1+2*k2+2*k3+k4)/6 + w1

         #fwd euler version
         w2 = dt*((R-1)*w1-w1**3)+w1

         w0 = np.real(sp.fft.ifft2(np.exp(A*.5*dt)*sp.fft.fft2(w2)))
     return w0
    \end{verbatim}

\section{Comparison with Chebfun}
\subsection{Chebfun vs Python 1}
\textbf{Chebfun 1 code}
\begin{verbatim}
   >> dom = [-16 16 -16 16]
   >> tspan = [0 10];
   >> S = spinop2(dom, tspan)
   >> S.lin = @(u) -2*lap(u)-biharm(u)
   >> R = .5
   >> S.nonlin = @(u) (R-1)*u - u.^3;
   >> u0 = 1/20*chebfun2(@(x,y) cos(x) + sin(2*x) + sin(y) + cos(2*y), dom, 'trig');
   u0 = u0 + chebfun2(@(x,y) exp(-((x-5*pi).^2 + (y-5*pi).^2)), dom, 'trig');
   u0 = u0 + chebfun2(@(x,y) exp(-((x-5*pi).^2 + (y-15*pi).^2)), dom, 'trig');
   u0 = u0 + chebfun2(@(x,y) exp(-((x-15*pi).^2 + (y-15*pi).^2)), dom, 'trig');
   u0 = u0 + chebfun2(@(x,y) exp(-((x-15*pi).^2 + (y-5*pi).^2)), dom, 'trig');
   u0 = u0 + chebfun2(@(x,y) exp(-((x-10*pi).^2 + (y-10*pi).^2)), dom, 'trig');
   S.init = u0;
   >> u = spin2(S, 256, .1, 'plot', 'off');
   >> plot(u), view(0,90), axis equal, axis off
   >> saveas(gcf,'/Users/edwardmcdugald/Research/
   convection_patterns_matlab/figs/sh_tsts_1018/cf1.pdf');
\end{verbatim}
% \begin{figure}[ht]
 %       \centering
  %      \includegraphics[scale=.7]{cf1.pdf}
   %     \caption{Chebfun SH 1}
   % \end{figure}
%\FloatBarrier

\textbf{python 1 code}
\begin{verbatim}
x = np.linspace(-16,16,256)
y = np.linspace(-16,16,256)
X,Y = np.meshgrid(x,y)
w0 = (1./20.)*(np.cos(X)+np.sin(2*X)+np.sin(Y)+np.cos(2*Y))
w0 = w0 + (np.exp(-((X-5*np.pi)**2 +(Y-5*np.pi)**2)))
w0 = w0 + (np.exp(-((X-5*np.pi)**2 +(Y-15*np.pi)**2)))
w0 = w0 + (np.exp(-((X-15*np.pi)**2 +(Y-15*np.pi)**2)))
w0 = w0 + (np.exp(-((X-15*np.pi)**2 +(Y-5*np.pi)**2)))
w0 = w0 + (np.exp(-((X-10*np.pi)**2 +(Y-10*np.pi)**2)))
dt = .1
R=.5
L=x[len(x)-1]-x[0]
nSteps = 100
W1 = integrateSH(w0,R,dt,nSteps,L)
fig1, ax1 = plt.subplots(nrows=1, ncols=1, figsize=(3,3))
ax1.imshow(W1)
plt.savefig("/Users/edwardmcdugald/Research/
convection_patterns/code/figs/sh_num_tsts_1018/mySH1.pdf")
\end{verbatim}
% \begin{figure}[ht]
 %       \centering
  %      \includegraphics[scale=1.0]{mySH1.pdf}
   %     \caption{My SH 1}
   % \end{figure}
%\FloatBarrier

\begin{figure}
\centering
\parbox{7cm}{
\includegraphics[width=7cm]{cf1.pdf}
\caption{Chebfun}
\label{fig:2figsA}}
\qquad
\begin{minipage}{7cm}
\includegraphics[width=7cm]{mySH1.pdf}
\caption{Python}
\label{fig:2figsB}
\end{minipage}
\end{figure}

%\begin{figure}[h]
 %   \begin{minipage}[c]{0.2\linewidth}
  %      \centering
  %      \includegraphics{cf1.pdf}
  %      \caption{Chebfun}
   % \end{minipage}
   % \begin{minipage}[c]{0.2\linewidth}
   %     \centering
   %     \includegraphics{mySH1.pdf}
   %     \caption{Python}
   % \end{minipage}
   % \caption{Chebfun v Python 1}
%\end{figure}


\subsection{Chebfun vs Python 2}
\textbf{chebfun 2 code}
\begin{verbatim}
dom = [0 20*pi 0 20*pi];
tspan = [0 200];
S = spinop2(dom, tspan);
S.lin = @(u) -2*lap(u) - biharm(u);
r = 1e-2;
S.nonlin = @(u) (-1 + r)*u - u.^3;
u0 = 1/20*chebfun2(@(x,y) cos(x) + sin(2*x) + sin(y) + cos(2*y), dom, 'trig');
u0 = u0 + chebfun2(@(x,y) exp(-((x-5*pi).^2 + (y-5*pi).^2)), dom, 'trig');
u0 = u0 + chebfun2(@(x,y) exp(-((x-5*pi).^2 + (y-15*pi).^2)), dom, 'trig');
u0 = u0 + chebfun2(@(x,y) exp(-((x-15*pi).^2 + (y-15*pi).^2)), dom, 'trig');
u0 = u0 + chebfun2(@(x,y) exp(-((x-15*pi).^2 + (y-5*pi).^2)), dom, 'trig');
u0 = u0 + chebfun2(@(x,y) exp(-((x-10*pi).^2 + (y-10*pi).^2)), dom, 'trig');
S.init = u0;
plot(S.init), view(0,90), axis equal, axis off
u = spin2(S, 96, 2e-1, 'plot', 'off');
plot(u), view(0,90), axis equal, axis off
>> saveas(gcf,'/Users/edwardmcdugald/Research/
convection_patterns_matlab/figs/sh_tsts_1018/cf2.pdf');
\end{verbatim}
% \begin{figure}[ht]
 %       \centering
 %       \includegraphics[scale=.7]{cf2.pdf}
 %       \caption{Chebfun SH 2}
 %   \end{figure}
%\FloatBarrier
\textbf{python 2 code}
\begin{verbatim}
x = np.linspace(0,20*np.pi,96)
y = np.linspace(0,20*np.pi,96)
X,Y = np.meshgrid(x,y)
w0 = (1./20.)*(np.cos(X)+np.sin(2*X)+np.sin(Y)+np.cos(2*Y))
w0 = w0 + (np.exp(-((X-5*np.pi)**2 +(Y-5*np.pi)**2)))
w0 = w0 + (np.exp(-((X-5*np.pi)**2 +(Y-15*np.pi)**2)))
w0 = w0 + (np.exp(-((X-15*np.pi)**2 +(Y-15*np.pi)**2)))
w0 = w0 + (np.exp(-((X-15*np.pi)**2 +(Y-5*np.pi)**2)))
w0 = w0 + (np.exp(-((X-10*np.pi)**2 +(Y-10*np.pi)**2)))
fig, ax = plt.subplots(nrows=1, ncols=1, figsize=(3,3))
ax.imshow(w0)
dt = 2e-1
R= 1e-2
L=x[len(x)-1]-x[0]
nSteps = 1000
W2 = integrateSH(w0,R,dt,nSteps,L)
fig2, ax2 = plt.subplots(nrows=1, ncols=1, figsize=(3,3))
ax2.imshow(W2)
plt.savefig("/Users/edwardmcdugald/Research/
convection_patterns/code/figs/sh_num_tsts_1018/mySH2.pdf")
\end{verbatim}
% \begin{figure}[ht]
%        \centering
%        \includegraphics[scale=1.0]{mySH2.pdf}
%        \caption{My SH 2}
%    \end{figure}
%\FloatBarrier
\begin{figure}
\centering
\parbox{7cm}{
\includegraphics[width=7cm]{cf2.pdf}
\caption{Chebfun}
\label{fig:2figsA}}
\qquad
\begin{minipage}{7cm}
\includegraphics[width=7cm]{mySH2.pdf}
\caption{Python}
\label{fig:2figsB}
\end{minipage}
\end{figure}


\subsection{Chebfun vs Python 3}
\textbf{chebfun 3 code}
\begin{verbatim}
>> dom = [0 20*pi 0 20*pi];
tspan = [0 200];
S = spinop2(dom, tspan);
S.lin = @(u) -2*lap(u) - biharm(u);
>> u0 = 1/20*chebfun2(@(x,y) cos(x) + sin(2*x) + sin(y) + cos(2*y), dom, 'trig');
u0 = u0 + chebfun2(@(x,y) exp(-((x-5*pi).^2 + (y-5*pi).^2)), dom, 'trig');
u0 = u0 + chebfun2(@(x,y) exp(-((x-5*pi).^2 + (y-15*pi).^2)), dom, 'trig');
u0 = u0 + chebfun2(@(x,y) exp(-((x-15*pi).^2 + (y-15*pi).^2)), dom, 'trig');
u0 = u0 + chebfun2(@(x,y) exp(-((x-15*pi).^2 + (y-5*pi).^2)), dom, 'trig');
u0 = u0 + chebfun2(@(x,y) exp(-((x-10*pi).^2 + (y-10*pi).^2)), dom, 'trig');
S.init = u0;
>> r = 7e-1;
>> S.nonlin = @(u) (-1 + r)*u - u.^3;
>> u = spin2(S, 96, 2e-1, 'plot', 'off');
>> plot(u), view(0,90), axis equal, axis off
>> saveas(gcf,'/Users/edwardmcdugald/Research/
convection_patterns_matlab/figs/sh_tsts_1018/cf3.pdf');
\end{verbatim}
% \begin{figure}[ht]
%        \centering
%        \includegraphics[scale=.7]{cf3.pdf}
%        \caption{Chebfun SH 3}
%    \end{figure}
%\FloatBarrier
\textbf{python 3 code}
\begin{verbatim}
x = np.linspace(0,20*np.pi,96)
y = np.linspace(0,20*np.pi,96)
X,Y = np.meshgrid(x,y)
w0 = (1./20.)*(np.cos(X)+np.sin(2*X)+np.sin(Y)+np.cos(2*Y))
w0 = w0 + (np.exp(-((X-5*np.pi)**2 +(Y-5*np.pi)**2)))
w0 = w0 + (np.exp(-((X-5*np.pi)**2 +(Y-15*np.pi)**2)))
w0 = w0 + (np.exp(-((X-15*np.pi)**2 +(Y-15*np.pi)**2)))
w0 = w0 + (np.exp(-((X-15*np.pi)**2 +(Y-5*np.pi)**2)))
w0 = w0 + (np.exp(-((X-10*np.pi)**2 +(Y-10*np.pi)**2)))
dt = 2e-1
R= 7e-1
L=x[len(x)-1]-x[0]
nSteps = 1000
W3 = integrateSH(w0,R,dt,nSteps,L)
fig3, ax3 = plt.subplots(nrows=1, ncols=1, figsize=(3,3))
ax3.imshow(W3)
plt.savefig("/Users/edwardmcdugald/Research/
convection_patterns/code/figs/sh_num_tsts_1018/mySH3.pdf")
\end{verbatim}
 %\begin{figure}[ht]
 %       \centering
 %       \includegraphics[scale=1.0]{mySH3.pdf}
 %       \caption{My SH 3}
 %   \end{figure}
%\FloatBarrier
\begin{figure}
\centering
\parbox{7cm}{
\includegraphics[width=7cm]{cf3.pdf}
\caption{Chebfun}
\label{fig:2figsA}}
\qquad
\begin{minipage}{7cm}
\includegraphics[width=7cm]{mySH3.pdf}
\caption{Python}
\label{fig:2figsB}
\end{minipage}
\end{figure}

\section{Chebfun v Python 4}
\textbf{chebfun 4 code}
\begin{verbatim}
>> dom = [0 20*pi 0 20*pi];
tspan = [0 200];
>> S = spinop2(dom, tspan);
>> S.lin = @(u) -2*lap(u) - biharm(u);
>> u0 = 1/20*chebfun2(@(x,y) cos(x) + sin(2*x) + sin(y) + cos(2*y), dom, 'trig');
u0 = u0 + chebfun2(@(x,y) exp(-((x-5*pi).^2 + (y-5*pi).^2)), dom, 'trig');
u0 = u0 + chebfun2(@(x,y) exp(-((x-5*pi).^2 + (y-15*pi).^2)), dom, 'trig');
u0 = u0 + chebfun2(@(x,y) exp(-((x-15*pi).^2 + (y-15*pi).^2)), dom, 'trig');
u0 = u0 + chebfun2(@(x,y) exp(-((x-15*pi).^2 + (y-5*pi).^2)), dom, 'trig');
u0 = u0 + chebfun2(@(x,y) exp(-((x-10*pi).^2 + (y-10*pi).^2)), dom, 'trig');
S.init = u0;
>> r = 1e-1;
>> S.nonlin = @(u) (-1 + r)*u - u.^3;
>> u = spin2(S, 100, 2e-1, 'plot', 'off');
plot(u), view(0,90), axis equal, axis off
>> saveas(gcf,'/Users/edwardmcdugald/Research/
convection_patterns_matlab/figs/sh_tsts_1018/cf4.pdf');
\end{verbatim}

\textbf{python 4 code}
\begin{verbatim}
x = np.linspace(0,20*np.pi,100)
y = np.linspace(0,20*np.pi,100)
X,Y = np.meshgrid(x,y)
w0 = (1./20.)*(np.cos(X)+np.sin(2*X)+np.sin(Y)+np.cos(2*Y))
w0 = w0 + (np.exp(-((X-5*np.pi)**2 +(Y-5*np.pi)**2)))
w0 = w0 + (np.exp(-((X-5*np.pi)**2 +(Y-15*np.pi)**2)))
w0 = w0 + (np.exp(-((X-15*np.pi)**2 +(Y-15*np.pi)**2)))
w0 = w0 + (np.exp(-((X-15*np.pi)**2 +(Y-5*np.pi)**2)))
w0 = w0 + (np.exp(-((X-10*np.pi)**2 +(Y-10*np.pi)**2)))
dt = 2e-1
R= 1e-1
L=x[len(x)-1]-x[0]
nSteps = 1000
W4 = integrateSH(w0,R,dt,nSteps,L)
fig4, ax4 = plt.subplots(nrows=1, ncols=1, figsize=(3,3))
ax4.imshow(W4)
plt.savefig("/Users/edwardmcdugald/Research/convection_patterns/code/figs/sh_num_tsts_1018/mySH4_1.pdf")

nSteps = 10000
W4b = integrateSH(w0,R,dt,nSteps,L)
fig4b, ax4b = plt.subplots(nrows=1, ncols=1, figsize=(3,3))
ax4b.imshow(W4b)
plt.savefig("/Users/edwardmcdugald/Research/convection_patterns/code/figs/sh_num_tsts_1018/mySH4_2.pdf")
\end{verbatim}
\begin{figure}
\centering
\parbox{7cm}{
\includegraphics[width=7cm]{cf4.pdf}
\caption{Chebfun}
\label{fig:2figsA}}
\qquad
\begin{minipage}{7cm}
\includegraphics[width=7cm]{mySH4_1.pdf}
\caption{Python (using 1000 steps)}
\label{fig:2figsB}
\end{minipage}
\end{figure}

\begin{figure}
\centering
\parbox{7cm}{
\includegraphics[width=7cm]{cf4.pdf}
\caption{Chebfun}
\label{fig:2figsA}}
\qquad
\begin{minipage}{7cm}
\includegraphics[width=7cm]{mySH4_2.pdf}
\caption{Python (using 10000 steps)}
\label{fig:2figsB}
\end{minipage}
\end{figure}

\textbf{ToDo: Compare Results with different initial value function}
\textbf{ToDo: Compare results using RK4 instead of Fwd Euler}
\textbf{ToDo: Add results of ellipse (with sigmoid)}


\end{document}

