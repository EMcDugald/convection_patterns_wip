\documentclass[12pt]{article}
\usepackage{amsmath}
\usepackage{amsthm}
\usepackage[hmargin=1in,vmargin=1in]{geometry}
\usepackage[shortlabels]{enumitem}
\usepackage{multicol}
\usepackage{hyperref}
\usepackage{graphicx}
\usepackage[section]{placeins}
\usepackage{amssymb}
\usepackage{bm}
\usepackage{listings}
\usepackage{amsmath,tkz-euclide}
\usepackage{tikz}
\usetikzlibrary{arrows.meta, decorations.markings}
\usetkzobj{all}
\newtheorem{theorem}{Theorem}
\newtheorem*{theorem*}{Theorem}
%\usepackage{float}
%\floatplacement{figure}{H}
\newcounter{problem}
\newcounter{solution}
\graphicspath{{/Users/edwardmcdugald/Research/convection_patterns/figs/}}
\usepackage{chngcntr}
\counterwithin*{section}{part}

\parindent 0in
\parskip 1em
\Urlmuskip=0mu plus 1mu

\title{Mathematical Analysis}
\author{Edward McDugald}


\begin{document}
\maketitle

\section{"Convection Patterns in Large Aspect Ratio Systems"}
\subsection{Introduction}
\begin{itemize}
    \item The Rayleigh Number is a dimensionless number associated with buoyancy-driven flow, aka free or natural convection. It characterizes the fluids flow regime- a lower value cooresponds to laminar flow, while a higher value corresponds to turbulent flow. Below a certain critical value, there is no fluid motion, and heat transfer is by conduction rather than convection. It is the product of the \emph{Grashof numner}, which describes the relationship between buoyancy and viscosity within a fluid. (Buoyancy=force opposing the weight of an immersed object, viscosity=resistance to deformation) and the \emph{Prandtl number}, which describes the relationship between momentum diffusivity and thermal diffusivity (momentum diffusivity=spread of momentum between particles, thermal diffusivity=thermal conductivity/density and specific heat capacity at constant pressure. Measures that rate of transfor of heat of a material from the hot end to the cold end). Small Prandtl numbers means thermal diffusivity dominates, whereas large prandtl numbers imply momentum diffusivity dominates.
    \item The general discussion is like this: There is work showing that a spatially uniform solution becomes unstable to a spatially singly periodic solution, ie transition from conduction to convection. This periodic state is stable for a range of Rayleigh numbers, prandtl numbers, and wave numbers (length of a pair of convecting rolls). But this analysis is on infinite domains. Observing convection experimentally on closed, finite domains reveals more complicated patterns. The goal is to explain these patterns. Qualitatively, they can be understood by orientational degeneracy, the band of stable wave numbers, and an observation that the rols tend to align themselves normal to the lateral boundaries.
    \item The idea to explain these patterns is to expand all quantities in a samll inhomogeneity parameter, and the method itself is analogous to WKB theory (see Strogatz lecture on youtube).
    \item Important observation: over much of the convecting field in a large box, a local wavevector may be defined. This wavevector varies slowly over the cell. We expect that there exist locally periodic solutions for the fluid variables $V(\theta, z;R)$, where $\nabla \theta = \vec{k}(x,y,t)$ is a slowly varying function. Ie, while the fluid variables vary over distances of the order of the roll size $d$, the wavevector $k$ varies over distances of the order of the lateral size of the box. The inverse aspect ratio $\eta^2$, the ratio of the cell depth $d$ to the lateral dimension $L$, is the small parameter that enters the theory. The Raylwigh number can be $\mathcal{O}(1)$ above the critical value $R_c$, but it must be less than the value at which the convection becomes unstable to disturbances with wavelengths of order $d$.
    \item The phase is a key object of the theory. We derive dynamic equations for the phase valid for slow variations of $\vec{k}$. This work is a generalization of the envelope equation theories by Newell and Whitehead, and of the phase diffusion equation introduced by Pomeau and Manneville to study small perturbations of straight roll patterns. Its key feature is that its rotational degeneracy is built into the formalism, and we can therefore describe the large reorientations of $\vec{k}$ over the box involved in the complicated patterns. Since the method assumes slow inhomogeneities, the solutions are limited to regions away from roll dislocations and other singular features. 
    \item Main achievement of this work is development of a canonical equation for the slow variation in both direction and magnitude of the wavevector of the convective pattern. Derivation relies upon the knowledge that (i) the spatially periodic motion exists and (ii) the spatially periodic motion is stable to local perturbations with wavevectors significantly different from the one locally present (\textbf{what does this statement mean?}). We know from the work of Busse on straight parallel rolls in the Oberbeck-Bousinessq equations that (i) and (ii) hold. The coefficients in the canoncial equation will depend on certain integrals of products of the underlying rapid field with itself and its derivatives. The fast description is very complicated, so we focus on models which bear a close resemblance to the oberbeck-bussinesq equations and whose initial instability shows all qualitative features of that system. We find that under certain weak assumptions, the resulting equation for the slow variables is universal in that it is independent of the details of the model. \textbf{some of these statements are bit murky}.
    \item The resulting equations contains all other theories to date. In particular, it reproduces all boundaries of the Busse balloon which border long wavelength instability regions, and thus the only patterns allowed by the system over times comparable with the horiozontal thermal diffusion time $L^2/\kappa$ must have local wavenumbers which lie within the balloon. For almost parallel rolls, this equation is the phase equation of Pomeau and Manneville. In the limit $R \rightarrow R_c$, the Newell-Whitehead-Segel envelope equations are recovered. 
    \item Some important features of the results: (i) the presence of "focus" singularities forces the roll wavenumber of approach to within $\mathcal{O}(\eta^2)$ of a unique value which lies at the left boundary of the Busse balloon and borders the region in which transverse perturbations grow. (ii) Such patterns should develop on the \emph{horizontal diffusion time scale}. We construct a function $F$ which under certain assumptions having to do with estimating the rolw that singularities play, acts as a Lyapunov functional. The decrease of this functional leads to $k$ values again within $\mathcal{O}(eta^2)$ of the left boundary of the Busse balloon. Thus, even though the original microscopic equations are not derivable from such a functional, the slow averaged equations, modulo singularities, are. (iii) Since the patches which are predicted to form on the horizontal diffusion time scale cannot tile the whole box, the system will not settle to any kind of equilibrium on this time scale. The subsequent motion must involve higher order terms in the phase equation which contain higher derivatives, or may involve dynamics on the rapid length scale, such as the pinching off of roles in the gliding of dislocation singularities between patches. A balance in the phase equation can be obtained by introducing a faster change, $\mathcal{O}(\eta^{-1})$ in the phase gradient along the rolls. This is allowed; since the wavenumber lies along the transverse instability boundary of the busse balloon, the rolls offer no resistance to this kind of bending. A quasi-universal equation is obtained, from which we derive similarity solutions which appear to describe the phase contours near single and miltiple dislocation defects. This also gives the prediction that the earliest time on which equilibrium could be reached is the horizontal diffusion time $L^2/\kappa$ multiplied by the aspect ration $L/d$.
    \item An important assumption in making these deductions is the empirical boundary condition that the rolls approach sidewalls at a perpendicular orientation to the wall.
    \item Note that analysis of full hydrodynamic equations is much more complicated. Even fully nonlinear spatially periodic solutions cannot be explicitly constructed. Further, due to the vertically uniform pressure field $P_s(x,y,t)$, a gradient expansion of Oberbeck-Boussinesq in terms of a single phase variable is not smooth, except at infinity Pramdtl number. Thus, the field $P_s$ should be incorporated into the expansion of $\theta_t$. The main conclusions are significantly changed by this. Thus, we consider model equations supplemented with mean drift terms. We also discuss the additional mechanisms for the onset of time dependence as the Rayleigh number is increased, in particular as it relates to the skew varicose instability.
  \end{itemize}
\subsection{Analysis}
\begin{itemize}
    \item The given models (I and II) are similar in that they both admit a simple class of one parameter, steady, periodic solutions.
        \[
            w(x,y)=Ae^{i\theta}, \theta = \bm{k}\cdot \bm{x}, \bm{x}=(x,y), \bm{k}=(m,n).
        \] 
    Here, model I is the complexified Swift-Hohenberg,
    \[
        \partial_t w + (\Delta+1)^2w-Rw+w^2w^{*}=0 \quad (1),
    \] 
    and we have
    \[
        R-A^2 = (k^2-1)^2 \quad (2).
    \] 
    \item The solutions correspond to straight, parallel rolls of fixed wave vector $\bm{k}$. They can point in any direction and exist for a range of $k$ determined by demanding $A^2>0$. This condition depends on $R$. The band of wavenumbers for which solutions exist also depends on $R$. One can show that patterns of superimposed rolls are unstable to single rolls. In reality, it is not likely that $\bm{k}$ is constant over the entire box (this is the "expected" single roll solution \textbf{does "single roll solution" mean $\bm{k}$ is constant?}). Due to orientational degeneracy \textbf{what is this exactly?}, the existence of a finite band of stable wavenumbers, and in particular the influence of the horizontal boundaries, one is led to look for \emph{locally} periodic solutions, in which $\bm{k}$ changes slowly but continuously over the box. 
    \item Introduce a fast phase,
        \[
            \theta(x,y,t)=\frac{1}{\eta^2}\Theta(X,Y,T),
        \] 
        with slow scales $X=\eta^2 x$, $Y=\eta^2 y$, $T=\eta^4 t$, where $\eta^2$ is the inverse aspect ratio, the ratio of roll size to box size. The local wavevcetor
        \[
            \bm{k}(X,Y,T)=(m,n)=(k\cos\psi, k\sin\psi)=\nabla_x\theta = \nabla_X\Theta,
        \] 
        is defined as the gradient of the phase. The relevant time scale is $1/\eta^4$, which is the horizontal diffusion time scale. We seek solutions of the two models (including the complexified Swift-Hohenberg) of the form
        \[
            w(x,y,t) = w^{(0)}(\theta; X,Y,T)+\sum_{p}\eta^{2p}w^{(p)}(\theta;X,Y,T) \quad (3),
        \] 
        where $w^{(0)}(\theta;X,Y,T)=f(\theta;A,k)=Ae^{i\theta}$. The parameters $A,k$ are no longer constant but functions of $X,Y,T$.
    \item This formulation has the advantage of establishing orientational degeneracy of the equations at the outset; no specific orientation is favored.
    \item Goal is to find equations for the quantities $A(X,Y,T)$ and $\bm{k}(X,Y,T)$. We do this by substituting (3) into (1),
\end{itemize}



\end{document}

