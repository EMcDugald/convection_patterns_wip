\documentclass[]{beamer}
\usepackage{beamerthemesplit}
\setbeamertemplate{caption}[numbered]
\usepackage{amsmath,bm}
\graphicspath{{/Users/edwardmcdugald/Research/convection_patterns_wip/figs/pres4/}}

\title{Finding a Universal Model of Nonlinear Pattern Forming PDEs in terms of the Phase Parameter}
\author{Edward McDugald}
\institute{University of Arizona}
\date{\today}
\begin{document}

\begin{frame}
  \titlepage
\end{frame}

\begin{frame}
    \frametitle{Motivation}
    \begin{itemize}
        \item We are interested in pattern forming systems, characterized by modulated stripe patterns. The canonical example of such patterns is Rayleigh-B\'{e}nard convection.
        \item Such patterns are observed as solutions to various non-linear PDEs. 
        \item However they are thought to all obey a universal diffusion equation in terms of the \emph{phase} of the pattern. We would like to derive such an equation.
     \end{itemize}
        \begin{figure}
        \centering
        \includegraphics[scale=0.3]{convection_roll_examples.png}
        \end{figure}
\end{frame}

\begin{frame}
     \frametitle{Cross-Newell Phase Diffusion Equation}
     \begin{itemize}
 \item Many nonlinear PDEs admit an exact straight roll solution of the form
     \[
         w = F(\theta=\vec{k}\cdot\vec{x},A) = \sum A_n(k)\cos{n\theta}.
     \] 
 \item For most natural patterns, the wave number $k=|\vec{k}|$ is not constant, but varies slowly over the box. So we seek a solution where the wave number is slowly modulating over the box
     \[
         w = F(\theta = \frac{1}{\epsilon}\Theta(\vec{X}=\epsilon\vec{x},T=\epsilon^2 t) = \frac{1}{\epsilon}\int \vec{k}d\vec{X}) + \epsilon w_1 + \dots.
     \] 
\item Using this ansatz, and imposing some restrictive assumptions, one can show that the phase $\Theta$ obeys the Cross-Newell equation
    \[
        \tau(k)\frac{\partial \Theta}{\partial T} = -\nabla \vec{k}B(k) - \epsilon^2 \eta \nabla^4 \Theta.
    \] 
     \end{itemize}
\end{frame}

\begin{frame}
     \frametitle{Cross-Newell as Regularization of a Diffusion Equation}
    \begin{itemize}
 \item The Cross-Newell equation can be written as
     \[
      \tau(k)\frac{\partial \Theta}{\partial T} = -\nabla^2\Theta B(k) - \epsilon^2 \eta \nabla^4 \Theta.
     \] 
  \item The quantities $B(k), (kB(k))'$ determine whether or not $\tau(k)\frac{\partial \Theta}{\partial T} = -\nabla^2\Theta B(k) $ is ill-posed. It turns out that this is a proper diffusion equation only for $k_B < k < K_E$.
\item For most natural patterns, the preffered wave number $k$ is something like $k_b-c$, ie, just to the left of $K_b$.
\item Thus, the biharmonic term is added to make the PDE well-posed.
    \end{itemize}
     \begin{figure}
        \centering
        \includegraphics[scale=0.5]{kbk.png}
      \end{figure}
\end{frame}

\begin{frame}
     \frametitle{Cross-Newell Equation as variation of Energy}
     \begin{itemize}
         \item Cross-Newell can be expressed as
             \begin{align*}
                 -\frac{\delta F}{\delta \Theta} &= \tau(k)\frac{\partial \Theta}{\partial T}\\
                 \epsilon F &= \int \left( 1 - |\nabla\Theta|^2\right)^2 d\vec{X} + \epsilon^2 \int \left(\nabla^2 \Theta\right)^2.
             \end{align*}
         \item The left term measures energy due to stretching, the right term measures energy due to bending.
        \item If we look for "self-dual" solutions by "balancing" the bending energy with the stretching energy, then such solutions solve the Cross-Newell equation if the Guassian curvature of the solution is 0.
        \item For a range of pattern defects/instabilities, one can assocaite a cost given by the energy functional.
     \end{itemize}
\end{frame}


\begin{frame}
     \frametitle{Bringing in some Machine Learning}
     \begin{itemize}
         \item The Cross-Newell equation is limited in the pattern defects it can predict.
        \item It's derivation dates back to the 80s, and there has been no proposed modification to capture more patterns.
        \item I want to use some data driven methods to analyze this problem.
        \item I am starting with simple methods for data driven discovery of PDEs.
        \item I will learn how to "rediscover" known non-linear PDEs that generate patterns of interet, and then concoct an experiment to propose governing equations of the corresponding phase surfaces.
     \end{itemize}
\end{frame}

\begin{frame}
    \frametitle{PDE Discovery via Sparse Optimization}
\begin{itemize}
  \item I combined ideas from Kutz et. al. and Hayden Schaeffer for finding governing equations via sparse regression.
  \item Suppose we have access to solutions $u(x,y,t)$ on a spatio-temporal grid, and in addition, we are able to estimate time derivatives for each solution, $u_t(x,y,t)$. Let us collect the time derivatives one vector, $U_t$.
\item 
    We can then form a feature matrix of tall, skinny feature vectors:
    \[
        F_u(t) = \begin{bmatrix}| & | & | & | & | & | & | & | & \dots\\
            1 & u & u^2 & u_x & u_x^2 & uu_x & u_xx & u_xx & \dots\\
                                | & | & | & | & | & | & | & | & \dots
                 \end{bmatrix}
    \] 
\item 
    We then approximate a solution of
    \[
        \xi = \text{argmin}\|F_u(t)\xi - U_t\|_2^2 + \lambda \|\xi\|_0
    \] 
\end{itemize}
\end{frame}

\begin{frame}
     \frametitle{"Discovering" Swift Hohenberg}
     \begin{itemize}
         \item I used Operator Splitting and Exponential Time Differencing to obtain numerical solutions of Swift-Hohenberg $u_t = -\Delta^2 u - 2\Delta u + (R-1)u - u^3.$ 
        \item I used backward finite differences for my time derivatives, and spectral derivatives for all spatial derivatives.
        \item I used a "coarse" grid for sampling. Ie, I took 20 time steps, and down sampled the spatial grids by a factor of 8.
        \item I used Sequentially Thresholded Ridge Regression. This is an iterative method which solve $\xi = \text{argmin}\|F_u(t)\xi-U_t\|_2^2 + \lambda \|\xi\|_2^2$ at each step, and applies a "hard" threshold to coefficients below a certain size, before repeating. The threshold tolerance is determined empirically by using an 80/20 test/train split. We used $\lambda = 1e-5$.
     \end{itemize}
\end{frame}

\begin{frame}
     \frametitle{"Discovering" Swift Hohenberg cont.}
     \begin{itemize}
         \item \href{https://github.com/EMcDugald/convection_patterns_wip/blob/master/code/ml_experiments/my_experiments/phase_surface.ipynb}{Notebook Demo}
     \end{itemize}
      \begin{figure}
        \centering
        \includegraphics[scale=0.25]{SH2.png}
        \end{figure}
 \begin{figure}
        \centering
        \includegraphics[scale=0.25]{STR.png}
        \end{figure}
\end{frame}

\begin{frame}
     \frametitle{Next Steps}
     \begin{itemize}
         \item Having successfully "rediscovered" Swift-Hohenberg, I am ready to extend the method to analysze the Cross-Newell Equation.
        \item A simple first experiment would be to generate Swift-Hohenberg solutions that obey the restrictions imposed in the Cross-Newell derivation. If I can generate phase surfaces from the solutions, I can run the experiment on the phase surfaces, and see if such an experiment predicts Cross-Newell.
        \item It would be neat to derive Cross-Newell, or a variant of it, directly from termperature surface simulations. I am wondering if there is a way to combine a neural network, or other machine learning technique to make this work.
        \item First step is to find a reliable method for estimating phase surfaces, and return to the mathematical analysis to understand the Cross-Newell limitations.
     \end{itemize}
\end{frame}

\begin{frame}
     \frametitle{Phase Surface Depiction}
      \begin{figure}
        \centering
        \includegraphics[scale=0.5]{SH3.png}
        \end{figure}
\end{frame}







\begin{frame}
    \frametitle{References}
    \begin{itemize}
    \item \href{https://reader.elsevier.com/reader/sd/pii/0167278984901817?token=A7642367BF58BF047EA0B36FF9405D2765AECE238093AEE2121CD36F6E79244B94C8B9C1B05FEAC38C9E9A6C6C6C37DC&originRegion=us-east-1&originCreation=20221012224737}{MC Cross and Alan C Newell. Convection patterns in large aspect ratio systems. 1984}
    \item \href{https://www.cmu.edu/cee/convergence/preprints/UniversalBehaviorModulatedStripePatterns.pdf}{Alan C. Newell and Shankar C. Venkataramani. The universal behavior of modulated stripe patterns 2022}
    \item \href{https://www.science.org/doi/pdf/10.1126/sciadv.1602614}{Kutz Et. Al. Data-driven discovery of partial differential equations. 2017}
    \item \href{https://royalsocietypublishing.org/doi/pdf/10.1098/rspa.2016.0446}{Hayden Schaeffer Learning partial differential equations via data discovery and sparse optimization, 2017}
        \item \href{https://github.com/EMcDugald/convection_patterns}{My Codes}
    \end{itemize}
\end{frame}










\end{document}
