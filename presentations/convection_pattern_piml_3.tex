\documentclass[]{beamer}
\usepackage{beamerthemesplit}
\setbeamertemplate{caption}[numbered]
\usepackage{amsmath,bm}
\graphicspath{{/Users/edwardmcdugald/Research/convection_patterns/figs/pres3/}}

\title{Data Driven Model Discovery for the Universal Behavior of Modulated Stripe Patterns}
\author{Edward McDugald}
\institute{University of Arizona}
\date{\today}
\begin{document}

\begin{frame}
  \titlepage
\end{frame}

\begin{frame}
    \frametitle{Motivation- Rayleigh-B\'{e}nard Convection in Experiments}
    \begin{itemize}
        \item Rayleigh-B\'{e}nard convection is observed by trapping a thin layer of fluid between two plates, and heating the bottom plate.
        \item Let $R$ denote the difference in temperature from the top plate to the bottom plate. Initially heat transfer occurs via \emph{conduction}. There is a critical value $R_c$ where heat transfer is a result of \emph{convection}. 
        \item When \emph{convection} arises, heat is moving due to motion of the fluid, and so called \emph{convection rolls} are observed.
        \item Research goal is to find a model of these patterns, capable of capturing all qualitatively distinct pattern arrangements and pattern defects.
    \end{itemize}
\end{frame}

\begin{frame}
    \frametitle{Convection Patterns- Experimental and Numerical Realizations}

    \begin{figure}
        \centering
        \includegraphics[scale=0.5]{convection_roll_examples.png}
        \end{figure}
\end{frame}

\begin{frame}
    \frametitle{Discrepancy Between Idealized Solution and Experimental Observations}
    \begin{itemize}
        \item In an idealized scenario, where the layer of fluid is infinite, there is a range of \emph{wave numbers} (length of the convection roll), and a range of Rayleigh numbers, for which there is a stable straight roll solution. This solution is rotationally and translationally invariant. (Galerkin)
        \item In experiments, there is obviously no "infinite" box. However, when heat is initally applied, the velocity and temperature field on one side of the box is independent of the velocity and temperature field on the other side of the box. Therefore, striped patterns emerge with different orientations in different regions of the box, and this breaks translational invariance.
        \item We want to understand how this mosaic of patches of striped patterns interact and form new patterns, as well as pattern defects.
    \end{itemize}
\end{frame}

\begin{frame}
    \frametitle{Initial Work to Model the Patterns}
    \begin{itemize}
        \item It is important to note that the stripe patterns observed in convection are thought to be an instance of a more general class of pattern forming systems.
        \item There are multiple PDEs whose solutions display patterns consistent with those seen in convection experiments.
        \item The distinguishing feature of such solutions is the presence of a striped pattern, whose periodicity, is \emph{slowly} varying over the box. 
        \item Work by M.C. Cross and Alan C. Newell resulted in a Phase Diffusion Equation to describe patterns which arise as modulated striped patterns.
    \end{itemize}
\end{frame}

\begin{frame}
    \frametitle{Local Periodicity of Convection Rolls}
      \begin{figure}
        \centering
        \includegraphics[scale=0.4]{cr_phase.png}
    \end{figure}
\end{frame}


\begin{frame}
    \frametitle{Derivation of Phase Diffusion Equation}
    \begin{itemize}
        \item The main idea is to study the phase parameter of the pattern. Under ideal conditions, the solution is periodic. In reality, the solution is \emph{locally} periodic. This means that the phase changes in time and space. 
        \item Most of the time, the phase changes \emph{slowly}. However, near defects, where striped patterns of different orientations collide, the phase changes \emph{rapidly}.
        \item So we introduce a slow time scale and a fast time scale.
    \end{itemize}
\end{frame}

\begin{frame}
    \frametitle{Derivation of Phase Diffusion Equation - Cont.}
    \begin{itemize}
        \item Start with Swift Hohenberg PDE
            \[
                \partial_t w + (\nabla^2+1)^2w - Rw + w^2w^{*} = 0 \quad (1)
            \] 
        \item A solution is given by 
            \[
                w(x,y)=Ae^{i\theta}, \theta= \bm{k} \cdot \bm{x}
            \] 
            as long as $R-A^2 = (k^2-1)^2$.
        \item Introduce a fast phase,
            \[
                \theta(x,y,t) = \frac{1}{\eta^2}\Theta(X,Y,T),
            \] 
            with $X=\epsilon x$, $Y=\epsilon Y$, $T=\epsilon^2 t$, $\epsilon = d/L$.
        \item Seek solutions to $(1)$ of the form
            \[
                w(x,y,t) = w^{(0)}(\theta;X,Y,T)+\sum_{p}\epsilon^{p}w^{(p)}(\theta;X,Y,T).
            \] 
    \end{itemize}
\end{frame}

\begin{frame}
    \frametitle{Derivation of Phase Diffusion Equation - Cont.}
    \begin{itemize}
        \item One obtains solvability conditions for $\Theta$, and arrives at the Cross-Newell phase diffusion equation
            \[
                \tau(k^2)\Theta_T + \nabla \cdot \bm{k}B(k^2)+\epsilon^2\eta\nabla^4\Theta = 0 \quad (2).
            \] 
    \end{itemize}
    \begin{figure}
        \centering
        \includegraphics[scale=0.5]{kbk.png}
    \end{figure}
\end{frame}

\begin{frame}
    \frametitle{Incorporating Machine Learning}
    \begin{itemize}
        \item Analysis to date has used theory from PDEs, Asymptotics, Variational Calculus, Topology and Geometry.
        \item I want to introduce a machine learning approach to solving the problem.
        \item My plan is to balance my time between understanding the mathematical analysis done on this problem, with conducting machine learning experiments for model discovery.
    \end{itemize}
\end{frame}

\begin{frame}
    \frametitle{SINDy- Sparse Identification of Non-Linear Dynamics}
    \begin{itemize}
        \item SINDy is a method described in 2016 by Kutz et. al. to extract parsimonious models from physical data.
        \item Casts the data discovery problem as a sparse regression, where data for the underlying system is available, and its derivatives are either available, or a suitable method is used to compute them.
        \item A matrix representing a library of functions is introduced, and a sparse regression is performed, resulting in a linear combination of the fewest terms of the library possible.
        \item Original work is only applicable to systems of differential equations.
    \end{itemize}
\end{frame}

\begin{frame}
    \frametitle{Original SINDy- Schematic}
     \begin{figure}
        \centering
        \includegraphics[scale=0.4]{orig_sindy.png}
    \end{figure}
\end{frame}

\begin{frame}
    \frametitle{Extending SINDy to PDEs}
    \begin{itemize}
        \item Kutz et. al. extended SINDy to be applicable to PDE Data.
     \end{itemize}
          \begin{figure}
        \centering
        \includegraphics[scale=0.4]{PDEFind.png}
    \end{figure}
\end{frame}

\begin{frame}
    \frametitle{Testing PDEFind on Reaction Diffusion Equation}
    I tested PDEFind on the following system:
    \begin{align*}
    u_t &= \left(1-(u^2+v^2) \right)u + \beta(u^2+v^2)v + d_1(u_{xx}+u_{yy})\\
    v_t &= -\beta(u^2+v^2)u + \left(1-(u^2+v^2)\right)v + d_2(v_{xx}+v_{yy}),
    \end{align*}
with $d_1,d_2=.1$,$\beta = 1$, and initial condition
\begin{align*}
    u(y_1,y_2,0)&= \tanh\left(\sqrt{y_1^2+y_2^2}\cos\left(\text{arg}(y_1+iy_2)-\sqrt{y_1^2+y_2^2}\right)\right)\\
    v(y_1,y_2,0) &= \tanh\left(\sqrt{y_1^2+y_2^2}\sin\left(\text{arg}(y_1+iy_2)-\sqrt{y_1^2+y_2^2}\right)\right).
\end{align*}
\end{frame}
\begin{frame}
    \frametitle{Reaction Diffusion Data Generation}
    \begin{itemize}
        \item The system can be solved by taking a Fourier Transform, and integrating the resulting system of ODEs. When done, take the inverse Fourier Transform at each time step.
        \item I solved on a $512 \times 512$ grid discretizing $[-10,10]^2$, for $200$ time steps in $t=[0,10]$. 
        \item \textbf{Link:}  \href{https://github.com/EMcDugald/convection_patterns/blob/master/code/scripts/solve_reaction_diffusion_champion.py}{RD Data Generation}
    \end{itemize}
 \begin{figure}
        \centering
        \includegraphics[scale=0.35]{rd_numerics.png}
    \end{figure}

\end{frame}
\begin{frame}
    \frametitle{PDEFind Prediction}
    \textbf{Link:}  \href{https://github.com/EMcDugald/convection_patterns/blob/master/code/PDEFIND/reaction_diffusion_v2.ipynb}{PDEFind Example}
 \begin{figure}
        \centering
        \includegraphics[scale=0.4]{PDEFind_prediction.png}
    \end{figure}
\end{frame}

\begin{frame}
    \frametitle{Beyond Sparse Regression}
    SINDy can be combined with an Autoeconder Neural Network to discover a coordinate system suitable to express sparse dynamics. I attempted to use this framework, but training took longer than expected. \textbf{Link:}  \href{https://github.com/EMcDugald/convection_patterns/tree/master/code/champion_codes}{Updating TF V1 to V2}
 \begin{figure}
        \centering
        \includegraphics[scale=0.4]{champion_schmatic.png}
    \end{figure}
\end{frame}

\begin{frame}
    \frametitle{SINDy and its modifications are available as an open source project- PySINDy}
    \href{https://github.com/dynamicslab/pysindy}{PySINDy Repo}
     \begin{figure}
        \centering
        \includegraphics[scale=0.4]{PySINDydemo.png}
    \end{figure}
 \end{frame}

 \begin{frame}
    \frametitle{Next Steps}
    \begin{itemize}
        \item I want to verify that PDEFind (as implemented in PySINDy) can identify the Swift-Hohenberg Equation.
        \item I have a numerical solver for Swift-Hohenberg, and can generate data quickly. \textbf{Link:} \href{https://github.com/EMcDugald/convection_patterns/blob/master/code/main/mySH_v1.py}{SH Solver}
    \end{itemize}
    \begin{columns}[c]
        \begin{column}{.5\textwidth}
            \begin{figure}
                \centering
                \includegraphics[scale=0.4]{mySH_tst1.pdf}
            \end{figure}
        \end{column}
        \begin{column}{.5\textwidth}
            \begin{figure}
                \centering
                \includegraphics[scale=0.4]{mySH_tst2.pdf}
            \end{figure}
        \end{column}
    \end{columns}
 \end{frame}
\begin{frame}
    \frametitle{Next Steps - Cont.}
    \begin{itemize}
        \item Assuming PDEFind identifies Swfit-Hohenberg, I want to incorporate the Autoencoder with the sparse regression.
        \item I am hoping the SINDy + Autoencoder method can identify a model in terms of the phase parameter.
        \item I will discuss with my advisor ways to incoroporate knowledge of the system into the method, making it properly "Physics Informed".
        \item Also interested in suggestions, ie Bayesian Optimization, Two Point Correlations, etc.
        \item Of course, I will be continuing with the Asymptotic, PDE Theory, Topological, etc. analysis.
    \end{itemize}
\end{frame}

\begin{frame}
    \frametitle{References (papers)}
    \begin{itemize}
    \item \href{https://reader.elsevier.com/reader/sd/pii/0167278984901817?token=A7642367BF58BF047EA0B36FF9405D2765AECE238093AEE2121CD36F6E79244B94C8B9C1B05FEAC38C9E9A6C6C6C37DC&originRegion=us-east-1&originCreation=20221012224737}{MC Cross and Alan C Newell. Convection patterns in large aspect ratio systems. 1984}
    \item \href{https://www.cmu.edu/cee/convergence/preprints/UniversalBehaviorModulatedStripePatterns.pdf}{Alan C. Newell and Shankar C. Venkataramani. The universal behavior of modulated stripe patterns 2022}
    \item \href{https://www.pnas.org/doi/pdf/10.1073/pnas.1517384113}{Kutz Et. Al. Discovering governing equations from data by sparse identification of nonlinear dynamical systems. 2016}
    \item \href{https://www.science.org/doi/pdf/10.1126/sciadv.1602614}{Kutz Et. Al. Data-driven discovery of partial differential equations. 2017}
    \item \href{https://www.pnas.org/doi/pdf/10.1073/pnas.1906995116}{Champion Et. Al. Data-driven discovery of coordinates and governing equations}
    \item \href{https://arxiv.org/pdf/2111.08481.pdf)}{Kaptanoglu Et. Al. PySINDy: A complehensive python package for robust sparse identification. 2022}
    \end{itemize}
\end{frame}
\begin{frame}
    \frametitle{References (codes)}
    \begin{itemize}
    \item \href{https://github.com/kpchamp/SindyAutoencoders}{SINDy+AutoEncoder Repo}
    \item \href{https://github.com/snagcliffs/PDE-FIND}{SINDY for PDE/PDEFIND Repo}
    \item \href{https://github.com/dynamicslab/pysindy}{PySINDy Repo}
    \item \href{https://pysindy.readthedocs.io/en/latest/}{PySINDy Docs}
    \item \href{https://github.com/EMcDugald/convection_patterns}{My Codes}
    \end{itemize}
\end{frame}










\end{document}
