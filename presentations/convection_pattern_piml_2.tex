\documentclass[]{beamer}
\usepackage{beamerthemesplit}
\setbeamertemplate{caption}[numbered]
\usepackage{amsmath}
\graphicspath{{/Users/edwardmcdugald/Research/convection_patterns_matlab/figs/piml_pres2/}}

\title{Convection Patterns Update: PDE Solutions + ML = Profit?}
\author{Edward McDugald}
\institute{University of Arizona}
\date{\today}
\begin{document}

\begin{frame}
  \titlepage
\end{frame}

\begin{frame}
    \frametitle{Chebfun for Easy Data Generation- Swift Hohenberg}
    Recall, SH Reads
    \[w_t = -(1+\nabla^2)^2w + Rw - w^3\]
    \begin{columns}[c]
    \begin{column}{.5\textwidth}
        \begin{figure}
        \centering
        \includegraphics[scale=0.15]{sh1.png}
        \end{figure}
    \end{column}
    \begin{column}{.5\textwidth}
        \begin{figure}
        \centering
        \includegraphics[scale=0.15]{sh2.png}
        \end{figure}
    \end{column}
    \end{columns}
        \begin{columns}[c]
    \begin{column}{.5\textwidth}
        \begin{figure}
        \centering
        \includegraphics[scale=0.15]{sh3.png}
        \end{figure}
    \end{column}
    \begin{column}{.5\textwidth}
        \begin{figure}
        \centering
        \includegraphics[scale=0.15]{sh4.png}
        \end{figure}
    \end{column}
    \end{columns}
\end{frame}

\begin{frame}
    \frametitle{My attempt at numerical solution}
    \begin{itemize}
        \item After reviewing the literature, I decided to try using my own "naive approach".
        \item I formed the discretization
            \[
            w^{k+1} = \Delta t \left[ -(2\nabla^2+\nabla^4)w^k + (R-1)w^k-(w^k)^3\right]+w^k
            \] 
        \item I will handle tha Laplacian and Biharmonic terms using FFT
            \[
                (2\nabla^2+\nabla^4)w^{k} = \text{ifft}((2K_{\Delta}+K_{\Delta}^2)\text{fft}(w^{k})).
            \] 
    \end{itemize}
\end{frame}

\begin{frame}[fragile]
    \frametitle{My attempt at numerical solution - Matlab Code}
    \begin{figure}
        \centering
        \includegraphics[scale=0.5]{matlabcd.png}
    \end{figure}
\end{frame}

\begin{frame}
    \frametitle{Bonus- Ginzburg-Landau Simulations }
    Ginzburg-Landau Reads
    \[w_t = \Delta w + w -(1+1.5i)w|w|^2\]
    \begin{columns}[c]
    \begin{column}{.5\textwidth}
        \begin{figure}
        \centering
        \includegraphics[scale=0.15]{gl1.png}
        \end{figure}
    \end{column}
    \begin{column}{.5\textwidth}
        \begin{figure}
        \centering
        \includegraphics[scale=0.15]{gl2.png}
        \end{figure}
    \end{column}
    \end{columns}
        \begin{columns}[c]
    \begin{column}{.5\textwidth}
        \begin{figure}
        \centering
        \includegraphics[scale=0.15]{gl3.png}
        \end{figure}
    \end{column}
    \begin{column}{.5\textwidth}
        \begin{figure}
        \centering
        \includegraphics[scale=0.15]{gl4.png}
        \end{figure}
    \end{column}
    \end{columns}
\end{frame}

\end{document}
