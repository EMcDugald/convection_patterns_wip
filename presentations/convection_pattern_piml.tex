\documentclass[]{beamer}
\usepackage{beamerthemesplit}
\setbeamertemplate{caption}[numbered]
\usepackage{amsmath}
\graphicspath{{/Users/edwardmcdugald/Research/convection_patterns/figs/}}

\title{In Search of a Model that Mimics Rayleigh-Benard Convection}
\author{Edward McDugald}
\institute{University of Arizona}
\date{\today}
\begin{document}

\begin{frame}
  \titlepage
\end{frame}

\begin{frame}
    \frametitle{Pattern Forming Properties of Rayleigh-Benard Convection}

    \begin{columns}[c]
    \begin{column}{.5\textwidth}
        \begin{figure}
        \centering
        \includegraphics[scale=0.20]{rb_1.png}
        \end{figure}
    \end{column}
    \begin{column}{.5\textwidth}
        \begin{figure}
        \centering
        \includegraphics[scale=0.20]{rb_2.png}
        \end{figure}
    \end{column}
    \end{columns}
\begin{itemize}
      \item Experimentally, Rayleigh-Benard Convection is observed by trapping a thin section of fluid between two plates, and heating the bottom plate.
    \item There is a parameter, $R$, the Rayleigh number, with an associated critical value $R_c$. When $R>R_c$, the fluid is set in motion, and convection rolls emerge.
    \item Such convection rolls are known to form an array of patterns, as well as patten defects.
\end{itemize}
\end{frame}

\begin{frame}
    \frametitle{The Swift-Hohenberg Equation}
        \begin{figure}
        \centering
        \includegraphics[scale=0.20]{rb_3.png}
        \caption{Numerical Simulation of SH}
        \end{figure}
\begin{itemize}
      \item \[
              w_t = -(1+\Delta)^2w + Rw - w^3,
      \] 
      where $w:\mathbb{R}^2 \rightarrow R$ represents advected temperature of \emph{Boussinesq} equations.
  \item Swift-Hohenberg is known for its pattern forming behaviors, and replicates many of the patterns and pattern defects observed in experiments.
  \item We would like to have simpler models that capture the same pattern forming properties observed in experiments. 
\end{itemize}
\end{frame}

\begin{frame}
    \frametitle{The Cross-Newell (and Regularized Cross-Newell) Equations}
        \begin{figure}
        \centering
        \includegraphics[scale=0.15]{rb_4.png}
        \caption{Numerical Simulation of CN}
        \end{figure}
\begin{itemize}
      \item Cross-Newell: $\Theta_T - kD_{\perp}(k)\nabla \cdot \vec{k} - D_{\parallel}(k)\vec{k}\cdot \nabla k=0$.
      \item Regularized Cross-Newell: $\tau(k)\Theta_T + \nabla \cdot \vec{k}B(k)+\eta \epsilon^2\nabla^4\Theta = 0$
      \item CN and RCN are derived assuming \emph{the absence} of defects. (Or the absence of certain kinds of defects).
    \item The research goal is to find a way to modify CN/RCN in such a way that all defect types are possible.
\end{itemize}
\end{frame}

\begin{frame}
    \frametitle{Goal for the Semester}
    \begin{itemize}
        \item Use PIML to find a model that captures the defects observed in Swift-Hohenberg simulations
        \item I will work on finding a way to numerically simulate Swift-Hohenberg. I know this is possible, since I have seen the simulations. My advisor knows where to find the method.
        \item It is my hope, that as long as there is data, PIML will output something useful.
        \item In parallel, I want to understand the CN/RCN derivations.
    \end{itemize}
\end{frame}





\end{document}
